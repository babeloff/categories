
\documentclass{article}
\usepackage[utf8]{inputenc}
\usepackage{amssymb}

\begin{document}

There are a bunch of terms associated with categories.

\begin{itemize}
  \item "NameSpace"
  \item "Name"
  \item "Version"
  \item "Dependencies"
  \item "Objects"
  \item "Arrows"
  \item "Paths"
  \item "Validation rules"
  \item "Enumeration of family members"
  \item "Travel to near family members"
  \item "Path equations"
  \item "Functors"
  \item "Labels"
  \item "Diagrams"
\end{itemize}

\section{Free Elements}

We are developing a grammmar that has a free syntax.
A free syntax in which many elements are implicit.
In particular a category requires that every object have an identity arrow.
As such including those arrows in addition to objects would be redundant.

What are the things we should get for free?

For all object identified we can infer the existence of all products.



\begin{itemize}
  \item $id_A$ every object has an identity morphism
  \item every object has an identity morphism
\end{itemize}

\section{by Sets}


\section{by Paths}

Categories are all about structure so it seems odd that we should start with objects.
What is the main element of structure?
It is, of course, the path.
Is it possible to just start with paths?

Consider that a category is defined as a namespace.
That namespace may, when required, may be given an appropriate alias.
Individual elements of that required category have immutable names which
are themselves unique within their namespace.
In this example the unique identifiers may be taken as numerals.
Paths are allowed, interesting compositions of other paths.
Here are some examples:
{ 1 [4 4], 2 [2], 3 [1 2], 4 [2 5 7 9] }

From this list of paths we can infer a few things.
But, something is missing.
What is missing is a way to describe properties of paths.
Paths may be of a sort: open or closed.
An closed path is one that returns back to its start without change.
The most obvious of the closed paths is the identity.
Typically we say there is a collection of objects and each object is required
to have an identity morphism.
An alternate, structural, way to think of this is to say
that a closed path forms an object.
The notions of similarity can then be seen to be related to
various types of closed paths.

An open path, then, is a path which does not define an object.
The purpose of an open path is to describe some type of similarity
between paths, the most obvious similarity is equivalence.
As objects are implicit in closed paths, arrows are implicit
as elements of paths generally.

The composition operation can be overloaded in its various arities.
[] : construct an atomic object.
[ x ] : construct an object from the object given.
[ x y & zs ] : compose a path from the objects given.

We can see the generation of anonymous paths and objects.
The naming (labeling) of these paths and objects is not part
of the structure but provides a mechanism for attaching constraints.


\subsection{[] 0-arity}

The use of the composition operator with no arguments forms an unnamed object.

{:foo [] :bar [] :baz []}

\subsection{[x] 1-arity}

Normally the x object is a path identifier from which an object is formed.
If an object is provided it could be considered
a generation of the identity morphism on the object
or it could be considered a no-op.

{:foo [] :id-foo [:foo] :baz [:id-foo]}

In the case given :baz is not equal to :foo even though they are both objects.
:baz is the $id_foo$ arrow as an object.

\subsection{[x y & zs] n-arity}

This form always creates a path.
In the case where the parameters are objects anonymous arrows will be created.



\end{document}
